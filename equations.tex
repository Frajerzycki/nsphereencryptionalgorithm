\documentclass[11pt]{article}
\usepackage[T1]{fontenc}
\newcommand\tab[1][1cm]{\hspace*{#1}}
\newcommand{\floor}[1]{\lfloor #1 \rfloor}

\begin{document}
	\section{Uwagi:}
	Jako oznaczenie współrzędnej w \ensuremath{n}-tym wymiarze jest użyta notacja taka jak do ciągów (\ensuremath{m_n}).\\
	Niezależnie od wyboru wiadomości da się skonstruować \ensuremath{l} wymiarową hipersferę o takim promieniu (\ensuremath{\sqrt{\sum\limits_{i=0}^l m^2_i}}) by punkt \ensuremath{m} leżał na niej.\\
	Szyfrowanie zadziała jeśli \ensuremath{\exists n (p_{n\bmod{k}} \neq m_n)}.
	\section{Wyprowadzenie \ensuremath{t} ze wzoru:}
	\begin{description}
	\item m - punkt leżący na hipersferze
	\item p - punkt przez który i przez \ensuremath{m} zostanie przeprowadzona prosta, której będą wyliczone wspólnepunkty z hipersferą.
	\item k - liczba wymiarów przestrzeni, w której jest punkt \ensuremath{p}
	\item l - liczba wymiarów hipersfery
	\end{description}
	\ensuremath {
	\\
	\sum\limits_{i=0}^l (m_i + t(p_{i \bmod{k}} - m_i))^2 = \sum\limits_{i=0}^l m^2_i \tab /- \sum\limits_{i=0}^l m^2_i \\
	\sum\limits_{i=0}^l (m_i + t(p_{i \bmod{k}} - m_i))^2 - \sum\limits_{i=0}^l m^2_i = 0 \\
	\sum\limits_{i=0}^l (m^2_i + 2m_it(p_{i \bmod{k}} - m_i) + t^2(p_{i \bmod{k}} - m_i)^2) - \sum\limits_{i=0}^l m^2_i = 0 \\
	\sum\limits_{i=0}^l (2m_it(p_{i \bmod{k}} - m_i) + t^2(p_{i \bmod{k}} - m_i)^2) = 0 \tab /: t\\
	\sum\limits_{i=0}^l (2m_i(p_{i \bmod{k}} - m_i) + t(p_{i \bmod{k}} - m_i)^2) = 0 \tab /- \sum\limits_{i=0}^l (2m_i(p_{i \bmod{k}} - m_i) \\
	-2\sum\limits_{i=0}^l m_i(p_{i \bmod{k}} - m_i) = t\sum\limits_{i=0}^l (p_{i \bmod{k}} - m_i)^2 \tab /: \sum\limits_{i=0}^l (p_{i \bmod{k}} - m_i)^2 \\
	t = -2\frac{\sum\limits_{i=0}^l m_i(p_{i \bmod{k}} - m_i)}{\sum\limits_{i=0}^l (p_{i \bmod{k}} - m_i)^2}
	}
	\newpage
	\section{Szyfrowanie:}
	Do reszt z dzielenia dodawane jest \ensuremath{w_{n\bmod{k}}} na wypadek, gdy \ensuremath{a_n = 0} i by zamaskować promień hipersfery. \\
	Wejście:
	\begin{description}
	\item m - wiadmość
	\item p - pierwsza część klucza
	\item w - druga część klucza
	\item k - długość pierwszej części klucza
	\item d - długość drugiej części klucza
	\item l - indeks ostatniego elementu wiadomości
	\end{description}
	Wyjście:
	\begin{description}
	\item q - zaszyfrowana wiadomość
	\item b - mianownik do użycia przy odszyfrowywaniu
	\end{description} 
	\ensuremath {
		a_n = p_{n \bmod{k}} - m_n \\
		b = \sum\limits_{i=0}^l a_i^2 \\
		c = 2\sum\limits_{i=0}^l m_ia_i \\
		e_n = bm_n - a_nc \\
		q_{n_0} = \floor{\frac{e_n}{b}}  \\
		q_{n_1} = e_n  - q_{n_0}b + w_{n\bmod{d}} 
	}
	\newpage
	\section{Odszyfrowywanie:}
	Wejście:
	\begin{description}
	\item q - zaszyfrowana wiadomość
	\item b - mianownik do użycia przy odszyfrowywaniu
	\item p - pierwsza część klucza
	\item w - druga część klucza
	\item k - długość pierwszej części klucza
	\item d - długość drugiej części klucza
	\item l - indeks ostatniego elementu zaszyfrowanej wiadomości
	\end{description} 
	Wyjście:
	\begin{description}
	\item m - wiadmość
	\end{description}
	\ensuremath {
		e_n = bq_{n_0} + q_{n_1} - w_{n\bmod{d}} \\
		f_n = bp_{n \bmod{k}} - e_n \\
		g = \sum\limits_{i=0}^l f_i^2 \\
		h = 2\sum\limits_{i=0}^l e_if_i \\
		d = gb \\
		m_n = \frac{ge_n - f_nh}{d}
	}
\end{document}