\documentclass[11pt]{article}
\usepackage[T1]{fontenc}
\newcommand\tab[1][1cm]{\hspace*{#1}}

\begin{document}
	\section{Uwagi:}
	Jako oznaczenie współrzędnej w \ensuremath{n}-tym wymiarze jest użyta notacja taka jak do ciągów (\ensuremath{m_n}).\\
	Niezależnie od wyboru wiadomości da się skonstruować \ensuremath{l} wymiarową hipersferę o takim promieniu (\ensuremath{\sqrt{\sum\limits_{i=0}^l m^2_i}}) by punkt \ensuremath{m} leżał na niej.\\
	Szyfrowanie zadziała jeśli \ensuremath{\exists n (p_{n\bmod{k}} \neq m_n)}.
	\section{Wyprowadzenie \ensuremath{t} ze wzoru:}
	\begin{description}
	\item m - punkt leżący na hipersferze
	\item p - punkt przez który i przez \ensuremath{m} zostanie przeprowadzona prosta na drugi koniec hipersfery - zostanie obliczony punkt wspólny hipersfery i prostej.
	\item k - liczba wymiarów przestrzeni, w której jest punkt \ensuremath{p}
	\item l - indeks ostatniego wymiaru hipersfery
	\end{description}
	\ensuremath {
	\\
	\sum\limits_{i=0}^l (m_i + t(p_{i \bmod{k}} - m_i))^2 = \sum\limits_{i=0}^l m^2_i \tab /- \sum\limits_{i=0}^l m^2_i \\
	\sum\limits_{i=0}^l (m_i + t(p_{i \bmod{k}} - m_i))^2 - \sum\limits_{i=0}^l m^2_i = 0 \\
	\sum\limits_{i=0}^l (m^2_i + 2m_it(p_{i \bmod{k}} - m_i) + t^2(p_{i \bmod{k}} - m_i)^2) - \sum\limits_{i=0}^l m^2_i = 0 \\
	\sum\limits_{i=0}^l (2m_it(p_{i \bmod{k}} - m_i) + t^2(p_{i \bmod{k}} - m_i)^2) = 0 \tab /: t\\
	\sum\limits_{i=0}^l (2m_i(p_{i \bmod{k}} - m_i) + t(p_{i \bmod{k}} - m_i)^2) = 0 \tab /- \sum\limits_{i=0}^l (2m_i(p_{i \bmod{k}} - m_i) \\
	-2\sum\limits_{i=0}^l m_i(p_{i \bmod{k}} - m_i) = t\sum\limits_{i=0}^l (p_{i \bmod{k}} - m_i)^2 \tab /: \sum\limits_{i=0}^l (p_{i \bmod{k}} - m_i)^2 \\
	t = -2\frac{\sum\limits_{i=0}^l m_i(p_{i \bmod{k}} - m_i)}{\sum\limits_{i=0}^l (p_{i \bmod{k}} - m_i)^2}
	}
	\newpage
	\section{
	Szyfrowanie jako punkt wspólny \ensuremath{l} wymiarowej hipersfery o promieniu\ensuremath{\sqrt{\sum\limits_{i=0}^l m^2_i}} i prostej przechodzącej przez punkty \ensuremath{p} i \ensuremath{m}:
	}
	Punkt m w rezultacie długości promienia leży na hipersferze.
	Dzielenie \ensuremath{e_n} przez b zostanie pominięte i wykonane będzie dopiero przy odszyfrowywaniu, kiedy wiadomo będzie, że dzielenie odwróci wynik (zwróci z powrotem wiadomość), więc będzie on na pewno całkowity.\\
	Przy liczeniu \ensuremath{e_n} dodawane jest do wyniku \ensuremath{p_{n\bmod{k}}\bmod{k}} chociaż nie ma tego we wzorach, żeby zamaskować wynik właściwego działania względnie niewielką liczbą na wypadek, gdy \ensuremath{a_n = 0}.\\
	Wejście:
	\begin{description}
	\item m - wiadmość
	\item p - klucz
	\item k - długość klucza
	\item l - indeks ostatniego elementu wiadomości
	\end{description}
	Wyjście:
	\begin{description}
	\item e - zaszyfrowana wiadmość
	\item b - mianownik do użycia przy odszyfrowywaniu
	\end{description} 
	\ensuremath {
		a_n = p_{n \bmod{k}} - m_n \\
		b = \sum\limits_{i=0}^l a_i^2 \\
		c = 2\sum\limits_{i=0}^l m_ia_i \\
		e_n = bm_n - a_nc + p_{n\bmod{k}}\bmod{k}
	}
	\newpage
		\section{
	Odszyfrowywanie jako przeprowadzenie prostej przez punkty \ensuremath{\frac{e_n}{b}} i \ensuremath{p} z powrotem do punktu \ensuremath{m} (na drugi koniec \ensuremath{l} wymiarowej hipersfery względem \ensuremath{\frac{e_n}{b}}), którym jest wiadmość:
	}
	Punkt \ensuremath{m} w rezultacie długości promienia leży na hipersferze, a \ensuremath{\frac{e_n}{b}} jest punktem wspólnym prostej i hipersfery, więc też leży na hipersferze.\\
	Wejście:
	\begin{description}
	\item e - zaszyfrowana wiadmość
	\item b - mianownik do użycia przy odszyfrowywaniu
	\item p - klucz
	\item k - długość klucza
	\item l - indeks ostatniego elementu zaszyfrowanej wiadomości
	\end{description} 
	Wyjście:
	\begin{description}
	\item m - wiadmość
	\end{description}
	\ensuremath {
		q_n = e_n - p_{n\bmod{k}}\bmod{k} \\
		f_n = bp_{n \bmod{k}} - q_n \\
		g = \sum\limits_{i=0}^l f_i^2 \\
		h = 2\sum\limits_{i=0}^l q_if_i \\
		d = gb \\
		m_n = \frac{gq_n - f_nh}{d}
	}
\end{document}